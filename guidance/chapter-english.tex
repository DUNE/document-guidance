\chapter{English Standards for LBNF/DUNE Documents}
\label{ch:english}

This chapter provides guidelines on terminology, grammar, punctuation, and other general writing guidelines for use in LBNF/DUNE documents. 


%%%%%%%%%%%%%%%%%%%%%%%%%%%%%%%%%%%%%%%%%%%%%%%%%%%%%%%%%%%%%%%%%%%%
\section{DUNE and LBNF-related Terminology}
\label{sec:english-terminology}

Please see the file \verb|common/glossary.tex| for defined terms. To use terms defined there, you will typically use \verb|\dword{label}| in your text.  Variations include:

\begin{itemize}
\item \verb|\dword{label}| nominal term
\item \verb|\dwords{label}| plural term
\item \verb|\Dword{label}| capitalized term
\item \verb|\Dwords{label}| capitalized and plural term
\item \verb|\dshort{label}| force usage of the abbreviated term
\item \verb|\dlong{label}| force usage of the full term
\end{itemize}

For example, to refer to the photon detection system, a glossary term has been defined:

\verb|\newduneabbrev{pds}{PDS}{photon detection system}| \\
\verb|{The \gls{submodule} system sensitive to light produced in the LAr}|

Now, everyone can refer to this in the text as \verb|\dword{\pds}|, and it will compile as ``PDS.''  If the editors decide to change it to ``PD'' later, they can easily do it globally.

Not all of the following terms are in the glossary; those that are, are labeled as such (purposely starting each line with lower-case letter for clarity):
\begin{itemize}
\item antiwhatever not anti-whatever (e.g., antineutrino, antimatter...);
\item ``muon antineutrino,'' not ``antimuon neutrino'';
\item minimum ionizing particle: MIP (use \verb|\dword{mip}|);
\item beamline, not ``beam line'' or ``beam-line'' (use \verb|\dword{beamline}|);
\item ``dual-phase module,'' not ``double-phase module'' (use \verb|\dword{dp}|);
\item ``detector'' should be used for the entire ND or FD, not for a module;
\item use Far Detector or Near Detector with initial caps. (use \verb|\dword{fd}| or \verb|\dword{nd}|);
\item use ProtoDUNE not protoDUNE;
\item avoid ProtoDUNE jargon, e.g., ``APA-2'';
\item  ``Facility'' is a common noun, unless the word ``facility'' is in the name of the structure;
\item ``detector module'' = any \SI{10}{kt} portion of the \SI{40}{kt} FD (use \verb|\dword{detmodule}|);
\item for non-reference design, call it ``alternative,'' not ``alternate'' (moot for TDR?);
\item ``a TPC in a \SI{10}{kt} module,'' not ``a \SI{10}{kt}  TPC''; Remember \SI{1}{kt} is metric and defined as \SI{1}{Gg};
\item co-spokespersons not co-spokespeople;
\item ``universe'' in general, not ``Universe.''
\item computer program names: Use italics on first use per chapter. Exceptions, e.g., \textit{art}, should always be italicized to distinguish them from common English words, in this case ``art.''
\end{itemize}

%%%%%%%%%%%%%%%%%%%%%%%%%%%%%%%%%%%%%%%%%%%%%%%%%%%%%%%%%%%%%%%%%%%%
\section{Spelling, Voice, Grammar and Punctuation}
\label{sec:english-spelling}

Spelling, voice:
\begin{itemize}
\item Standard English rules for capitalization apply. Do not promote common nouns or adjectives to proper nouns or adjectives.
\item Use American spelling: e.g., ionization (not ionisation), flavor (not flavour) and so on.
\item In general, avoid use of first person (e.g., I, we, our).  ``We'' may appear in introductory sections.
\item Avoid use of second person, i.e., ``you.''
\item The gender-neutral ``singular they'' is to be used in preference to ``he'' or ``she'' or ``he/she'' or other ; e.g., ``the installer must wear a hard hat when they enter the cryostat.''
\item Use tables and figures to highlight content. It increases readability.
\item Use active voice wherever possible, e.g., ``the field shaping coils and the cathode structure share identical features...'' or ``the design follows  an extensive review...''
\item Use consistent terminology. (If you call it a spade here, call it a spade there.) Use DUNE Words as much as possible!
\item Expand abbreviations at first occurrence. Use DUNE Words (\url{https://dune.bnl.gov/docs/technical-proposal/dune-words.pdf})!
\item Do not split infinitives. E.g., ``to choose carefully'' is good, but ``to carefully choose'' is bad.
\item Avoid starting sentences with an acronym; spell out the term in this case.
\end{itemize}

Numbers:
\begin{itemize}
\item Generally, spell out numbers one through ten. Use numerals for \num{11} and above.
\item Avoid starting sentences with a numeral.
\item Use numerals when giving dimensions, weight, distance, etc. Other usages that always use numerals, for example, are dollars (\$1 billion, not \$\num{1000000000}), ratios, etc. Also tables use numerals exclusively (exceptions may exist).
\item Speaking of monetary amounts, symbols for currencies are acceptable, e.g., \verb|\euro \textsterling| \\
 \verb|(or \pounds) \textyen \rupee| to get \euro, \pounds, \textyen, \rupee{} respectively. The Brazilian real is written \verb|R\$| to get R\$.

\item Examples of adjacent numbers: seven one-ton-rated-or-less pickups ... two vans (one cargo, one equipment)... one eight-passenger suburban, one-hundred fifty \SI{7}{\meter} cables or 150 seven meter cables ...
\end{itemize}

Punctuation:
\begin{itemize}
\item Notice that ``e.g.,'' has two periods and a comma; so does ``i.e.,''.
\item Normally commas and periods go inside of quotations;  semi-colons and exclamation points go outside.  (In previous item, the commas are part of the terms in quotes.)
\item Minimize capitalization in captions.
\item Even if a normal English word is used as part of the proper name of an element of the experiment (part of the detector, a software package, etc.,) leave it in lower case. E.g., anode plane assembly. 
\item Do not pluralize abbreviations using an apostrophe (') following the abbreviation. Example: TDRs are important design documents. Not TDR's. 
\begin{itemize}
\item Exception: Use an apostrophe when the abbreviation or acronym has internal periods. 
\item Exception number \num{2}: Use nu sub e plural (and others) with apostrophe; e.g., many $\nu$'s do not make light work. You may use the full word, e.g., ``neutrino'' in this case.
\end{itemize}
\item Avoid single quotation marks ('). Best to use \textit{italics} to set off special references. You can use double quotes (``example''). 
\item Use ``Oxford commas'' (e.g., after the y in ``x, y, and z'') 
\end{itemize}


%%%%%%%%%%%%%%%%%%%%%%%%%%%%%%%%%%%%%%%%%%%%%%%%%%%%%%%%%%%%%%%%%%%%
\section{Abbreviations}
\label{sec:english-abbrevs}

\begin{itemize}
\item look for commonly used terms and abbreviations in common/defs.tex, and add to it as you see fit 
\item can use ND, FD for near and far detector
\item USA not US
\item UK not U.K. 
\item Ph.D not PhD.
\item ``a'' liquid argon TPC, and also ``a'' LArTPC (i.e., a ``lar'' TPC not an ``ell-ay-are-TPC'') 
\item \si{\kt} not \si{\kton} (kiloton only when used as in ``a kiloton of LAr is...'') 
\item Avoid the use of ``ton'' or ``tonne''; instead use \SI{1000}{\kilo\gram}. Also use the abbreviation \si{\kt} as called for.
\item Use LAr and LN (not LN2) for liquid argon and nitrogen, respectively. 
\item Avoid GAr or GN for gaseous argon and nitrogen, just write them out.
\end{itemize}

%%%%%%%%%%%%%%%%%%%%%%%%%%%%%%%%%%%%%%%%%%%%%%%%%%%%%%%%%%%%%%%%%%%%
\section{Hyphenation}
\label{sec:english-hyphen}

\begin{itemize}
\item to separate for emphasis: use two dashes together: \verb|--| in text, e.g.,  ``the system -- not to mention its three subsystems -- does such and such'' 
\item antiwhatever not anti-whatever (e.g., antineutrino, antimatter...)
\item beamline not ``beam line'' or ``beam-line'' 
\item ``long-baseline experiment''  (with hyphen) vs. ``long baseline of the experiment'' (without)
\item world-class but worldwide   
\item readout, not read-out (when used as a noun or adjective); e.g., ``the DAQ readout (system) supports...'' but ``in order to read out the DAQ...''
\item offline, not off-line
\item hyphen in double-beta decay, (and inverse-beta decay)
\item hyphen when first adjective modifies following adjective, not the noun. E.g. ``deep-underground location'' vs. ``located deep underground'';  this is not strictly necessary; use as you see fit for clarity.
\item no hyphens between number and unit (this is a change from CDR); i.e., an \SI{8}{\meter} pipe.
\item avoid use of the slash between words, use ``and'' or ``or''. (Only use and/or when really necessary).
\end{itemize}

%%%%%%%%%%%%%%%%%%%%%%%%%%%%%%%%%%%%%%%%%%%%%%%%%%%%%%%%%%%%%%%%%%%%
\section{Lists}
\label{sec:english-lists}

\begin{itemize}
\item In paragraphs that enumerate lists of things, use bullet or numbered lists (\textit{itemize} or \textit{enumerate} in LaTeX) unless it seems overblown.
\item Normally use bullets. Use numbers only if there is an implied order to your list; e.g., steps in a procedure, priority, etc.
\item If bullets seem overblown,  use open-close parens with number, e.g., (1) blah, (2) blahh, ... and (n) blahhhhh. I.e., do not use other formats like 1), 2), or (a), (b)... etc.
\item Use a stem sentence to introduce a list and write the stem in a complete sentence, wherever possible.
\item Avoid using the term ``list,'' ``listed,'' ``below,'' and so on, in stem sentences. It is redundant. 
\item Follow parallel construction in lists; e.g., all phrases, all single words, all sentences, all phrases beginning with an action verb, ...
\item Use the right punctuation, i.e., use periods after complete sentences. Use either commas or semicolons after phrases. No punctuation at all for lists of words. There can be variations on this.
\item Use consistent capitalization. Initial cap the first word in a list if it is a sentence or phrase. 
\item Use only up to two sub-levels of lists.
\item It is an oversight if some of the lists in this chapter don't follow all these rules!
\end{itemize}

%%%%%%%%%%%%%%%%%%%%%%%%%%%%%%%%%%%%%%%%%%%%%%%%%%%%%%%%%%%%%%%%%%%%
\section{Common Errors}
\label{sec:english-errors}

\begin{itemize}
\item Element names are not capitalized except when their symbols are used. I.e., ``argon'' and ``Ar''; ``gold'' and ``Au.''
\item Use the term ``comprise'' to mean ``include'' not ``compose.'' The word comprise means ``to be composed of.''
\item Verbiage does not mean ``text.'' It means wasted or superfluous text.
\item Enormity does not mean huge as in ``enormous.'' Enormity means evil.
\item Do not interchange the words ``affect'' and ``effect.'' Affect is used as a verb and effect is almost always used as a noun to mean the result of a cause. (As a verb it means ``to bring about'' as in ``to effect a change.'') 
\item Use ``impact'' as a noun and not as a verb to mean ``affect.''
\item Use of ``less'' and ``fewer'': Fewer is associated with quantity that can be counted. Less is associated with volume or mass. ``Fewer people have less mass after eating at McDonald's than have more mass.''
\item Use of ``that'' and ``which'': 
\begin{itemize}
\item Use ``that'' for restrictive clauses (clauses that cannot be removed without altering the meaning of the sentence). E.g., ``She uses the chair that goes with this desk.''
\item Use ``which'' for non-restrictive clauses (clauses that can be removed without altering the meaning of a sentence.) Tip: Use ``which'' in conjunction with a comma and check if that clause can be removed without changing the sense.  E.g., ``A desk like this, which would be too low for you, should have a chair like that.''
\end{itemize}
\item Use of the word ``concur'': You concur \emph{with} someone and you concur \emph{in} an idea or an opinion. 
\item The opposite holds true with the word ``disappoint.'' You are disappointed \emph{with} something and disappointed \emph{in} someone.
\end{itemize}



