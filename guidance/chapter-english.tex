\chapter{English standards for LBNF/DUNE documents}
\label{ch:english}

This chapter provides guidelines on terminology, grammar, punctuation, and other general writing guidelines for use in LBNF/DUNE documents. 


%%%%%%%%%%%%%%%%%%%%%%%%%%%%%%%%%%%%%%%%%%%%%%%%%%%%%%%%%%%%%%%%%%%%
\section{Spelling, voice, grammar and punctuation}
\label{sec:english-spelling}

\begin{itemize}
\item Use American spelling: e.g., ionization (not ionisation), flavor (not flavour) and so on.
\item Notice that ``e.g.,'' has two periods and a comma; so does ``i.e.,''.
\item Normally commas and periods go inside of quotations;  semi-colons and exclamation points go outside.  (In previous item, the commas are part of the terms in quotes.)
\item Use active voice as much as possible. 
\item Avoid use of first person (e.g., I and we). Use ``the team'' or ``the group,'' etc.
\item Use consistent terminology. (If you call it a spade here, call it a spade there.)
\item Expand abbreviations at first occurrence.
\item Generally, spell out numbers one through nine. Use numerals for \num{10} and above.
\item Use numerals when giving dimensions, weight, distance, etc. Other usages that always use numerals, for example, are dollars (\$1 billion, not \$\num{1000000000}), ratios, etc. Also tables often use numerals for everything.
\item For example: seven 1-ton-rated-or-less pickups ... two vans (one cargo, one equipment)... one eight-passenger suburban ...
\item Avoid starting sentences with a numeral.
\item Do not split infinitives. E.g., ``to choose carefully'' is good, but ``to carefully choose'' is bad.
\item Minimize capitalization in headings and captions.
\item Do not pluralize abbreviations using an apostrophe (') following the abbreviation. Example: CDRs are important design documents. Not CDR's. 
\begin{itemize}
\item Exception: Use an apostrophe when the abbreviation or acronym has internal periods. 
\item Exception number \num{2}: Use nu sub e plural (and others) with apostrophe; e.g., many $\nu$'s do not make light work.
\end{itemize}
\item Use of single quotation marks (') is non-standard. Best to use \emph{italics} to set off special references. Can use double quotes (``example''). 
\item Use tables and figures to highlight content. It increases readability. 
\end{itemize}



%%%%%%%%%%%%%%%%%%%%%%%%%%%%%%%%%%%%%%%%%%%%%%%%%%%%%%%%%%%%%%%%%%%%
\section{Terminology}
\label{sec:english-terminology}

\begin{itemize}
\item ``universe'' in general, not ``Universe''
\item antiwhatever not anti-whatever (e.g., antineutrino, antimatter...)
\item ``muon antineutrino,'' not ``antimuon neutrino''
\item minimum ionizing particle: MIP
\item beamline, not ``beam line'' or ``beam-line''
\item ``dual-phase,'' not ``double-phase''
\item ``detector'' should be used for the entire ND or FD, not for a module
\item use Far Detector or Near Detector with initial caps when you refer to the name of a building such as Far Detector Hall or Near Detector Hall; otherwise ``far'' and ``near'' detector.
\item same for ``facility'' as it is a common noun, unless the word ``facility'' is in the name of the structure
\item ``detector module'' = any \ktadj{10} portion of the FD
\item reference FD is \SI{40}{kt}
\item for non-reference design, call it ``alternative,'' not ``alternate'' (moot for TDR?)
\item ``a TPC in a \ktadj{10} module,'' not ``a \ktadj{10}  TPC'' 
\item Use of ``that'' and ``which'': 
\begin{itemize}
\item Use ``that'' for restrictive clauses (clauses that cannot be removed without altering the meaning of the sentence). E.g., ``She uses the chair that goes with this desk.''
\item Use ``which'' for non-restrictive clauses (clauses that can be removed without altering the meaning of a sentence.) Tip: Use ``which'' in conjunction with a comma and check if that clause can be removed without changing the sense.  E.g., ``A desk like this, which would be too low for you, should have a chair like that.''
\end{itemize}
\end{itemize}

%%%%%%%%%%%%%%%%%%%%%%%%%%%%%%%%%%%%%%%%%%%%%%%%%%%%%%%%%%%%%%%%%%%%
\section{Abbreviations}
\label{sec:english-abbrevs}

\begin{itemize}
\item can use ND, FD for near and far detector
\item U.S. not US
\item Ph.D not PhD.
\item ``a'' liquid argon TPC, and also ``a'' LArTPC (i.e., a ``lar'' TPC not an ``ell-ay-are-TPC'') 
\item \si{kt} not \si{kton} (kiloton only when used as in ``a kiloton of LAr is...'') 

\end{itemize}

%%%%%%%%%%%%%%%%%%%%%%%%%%%%%%%%%%%%%%%%%%%%%%%%%%%%%%%%%%%%%%%%%%%%
\section{Hyphenation}
\label{sec:english-hyphen}

\begin{itemize}
\item to separate for emphasis: use two dashes together: \verb|--| in text, e.g.,  ``the system -- not to mention its three subsystems -- does such and such'' 
\item antiwhatever not anti-whatever (e.g., antineutrino, antimatter...)
\item beamline not ``beam line'' or ``beam-line'' 
\item world-class but worldwide   
\item readout, not read-out
\item offline, not off-line
\item co-spokespersons not co-spokespeople
\item dash in double-beta decay, (and inverse-beta decay)
\item I've been using a hyphen when first adjective modifies following adjective, not the noun. E.g. ``deep-underground location'' vs. ``located deep underground.''  This is not strictly necessary; use as you see fit.
\item ``a \ktadj{10} mass module'' (with hyphen) vs. ``module has mass of \SI{10}{kt}'' (without)
\item high-resolution detector (with hyphen) vs. high resolution of the detector (without)
\item may use ``Oxford commas'' (e.g., after the y in ``x, y, and z'')
\end{itemize}

%%%%%%%%%%%%%%%%%%%%%%%%%%%%%%%%%%%%%%%%%%%%%%%%%%%%%%%%%%%%%%%%%%%%
\section{Lists}
\label{sec:english-lists}

\begin{itemize}
\item In paragraphs that enumerate lists of things, use bullet or numbered lists ('itemize' or 'enumerate' in LaTeX) unless it seems overblown.
\item If bullets seem overblown,  use open-close parens with number, e.g., (1) blah, (2) blahh, ... and (n) blahhhhh. I.e., do not use other formats like 1), 2), or (a), (b)... etc.
\item Use numbers only if there is an implied order to your list; e.g., steps in a procedure, priority, etc.
\item Otherwise use bullets.
\item Use a stem sentence to introduce a list and write the stem in a complete sentence, wherever possible.
\item Avoid using the term ``list,'' ``listed,'' ``below,'' and so on, in stem sentences. It is redundant. 
\item Follow parallel construction in lists; e.g., all phrases, all single words, all sentences, all phrases beginning with an action verb, ...
\item Use the right punctuation, i.e., use periods after complete sentences. Use either commas or semicolons after phrases. No punctuation at all for lists of words. There can be variations on this.
\item Use consistent capitalization. Initial cap the first word in a list if it is a sentence or phrase. 
\item Use only up to two sub-levels of lists.
\item It is an oversight if some of the lists in this chapter don't follow all these rules!
\end{itemize}

%%%%%%%%%%%%%%%%%%%%%%%%%%%%%%%%%%%%%%%%%%%%%%%%%%%%%%%%%%%%%%%%%%%%
\section{Common errors}
\label{sec:english-errors}

\begin{itemize}
\item Use the term ``comprise'' to mean ``include'' not ``compose.'' The word comprise means ``to be composed of.''
\item Verbiage does not mean ``text.'' It means wasted or superfluous text.
\item Enormity does not mean huge as in ``enormous.'' Enormity means evil.
\item Do not interchange the words ``affect'' and ``effect.'' Affect is used as a verb and effect is almost always used as a noun to mean the result of a cause. (As a verb it means ``to bring about'' as in ``to effect a change.'') 
\item Use ``impact'' as a noun and not as a verb to mean ``affect.''
\item Use of ``less'' and ``fewer'': Fewer is associated with quantity that can be counted. Less is associated with volume or mass. ``Fewer people have less mass after eating at McDonald's than have more mass.''
\item Use of the word ``concur'': You concur \emph{with} someone and you concur \emph{in} an idea or an opinion. 
\item The opposite holds true with the word ``disappoint.'' You are disappointed \emph{with} something and disappointed \emph{in} someone.
\end{itemize}



