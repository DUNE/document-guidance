\chapter{Technical}

This chapter describes some of the more technical aspects of the CDR.

%%%%%%%%%%%%%%%%%%%%%%%%%%%%%%%%%%%%%%%%%%%%%%%%%%%%%%%%%%%%%%%%%%%%
\section{Getting the Files and Compiling after your Changes}
\label{sec:getfiles-compile}

Instructions for getting the files and compiling a volume are given on the home page of the repository, at
\href{https://github.com/LBNE/lbn-cdr}{https://github.com/LBNE/lbn-cdr}. Scroll down to 
``Guidance'' and ``Getting Started.''

Note that any acronyms list that a volume has requires an extra step to be included; this step
is not documented there.
The technical editors will take care of it.

%%%%%%%%%%%%%%%%%%%%%%%%%%%%%%%%%%%%%%%%%%%%%%%%%%%%%%%%%%%%%%%%%%%%
\section{The \LaTeX{} CDR class}

All of the \LaTeX{} configuration for the document that pertains to the general CDR style and not to the actual content is in the \texttt{cdr.cls} file.  The class takes some options that control high-level style:

\begin{description}
\item[\texttt{draft}] produce markup to assist in editing (line numbers, draft water mark, label tags)
\item[\texttt{print}] print quality (remove editing markup)
\end{description}

Instructions for producing one or the other are listed on the GitHub site referenced in Section~\ref{sec:getfiles-compile}.

%%%%%%%%%%%%%%%%%%%%%%%%%%%%%%%%%%%%%%%%%%%%%%%%%%%%%%%%%%%%%%%%%%%%
\section{Volume-generic \LaTeX{} files}

There are several files specific to the topic of the CDR but used for all the volumes, hence 
they are in the \texttt{common} directory:

\begin{description}
\item[\texttt{common/preamble.tex}] placed before the document begins and should provide all macros to define common terms or units.
\item[\texttt{common/init.tex}] placed immediately after the document starts and should provide things like title page, author list, toc/lof/lot, etc.
\item[\texttt{common/final.tex}] placed immediately before the document ends and should provide the bibliography setup or any other common trailing matter.
\item[\texttt{common/defs.tex}] (input to the preamble) contains macros for some commonly 
used terms and expressions. Please review this file and use the definitions to help ensure consistency
throughout the document.
\end{description}

Also in the \texttt{common} directory are two content-related files: \texttt{intro.tex} and \\
\texttt{supp-doc-list.tex}.  All volumes except the introductory volume will share the introduction
contained in \texttt{intro.tex}.  The other is a list of supporting documents that this
introduction file pulls in.

%%%%%%%%%%%%%%%%%%%%%%%%%%%%%%%%%%%%%%%%%%%%%%%%%%%%%%%%%%%%%%%%%%%%
\section{The volume-(name).tex file}

\hlfix{Contributors do not need to do this!} \\

To start a new volume, copy the \texttt{guidance.tex} to a new
name and edit as directed by the comments.  

\begin{enumerate}
\item Set the graphics path
\item Redefine the volumes sub-title
\item Input each chapter file.
\end{enumerate}

