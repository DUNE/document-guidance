\chapter{Important Files in the Document Repository}
\label{ch:tech}

This chapter describes some of the more technical aspects of the LBNF/DUNE \LaTeX setup. Various aspects of the configuration are pulled out into separate files to keep things well organized.  Some files are meant to be edited by contributors, others are only for the technical editing team; these latter are labeled ``Please do not edit this file'' below.

%%%%%%%%%%%%%%%%%%%%%%%%%%%%%%%%%%%%%%%%%%%%%%%%%%%%%%%%%%%%%%%%%%%%
\section{The Main File}
\label{sec:tech-mainfile}

A new main file will likely be started only by the technical editors. There is one for each volume (if a document is split into volumes). It is the file that you compile to create the document or the volume. It appears in the top level directory of the repository.  These files are named \texttt{volume-(something).tex} (or \texttt{(document-name).tex} if not multi-volume). They should be the only \texttt{(name).tex} files at this directory level.

The main file...

\begin{enumerate}
\item sets the document class (see Section~\ref{sec:tech-classfile};
\item sets the graphics path;
\item inputs the preamble;
\item opens the document environment;
\item inputs the ``init'' file;
\item Redefine the volumes subtitle when document is split into volumes
\item inputs each chapter file;
\item inputs the ``final'' file (which provides the bibliographical information); and
\item closes the document environment.
\end{enumerate}

To start a new document, you would copy the \texttt{guidance.tex} to a new
name and edit as directed by the comments.  As an example, the main file of the physics volume of the DUNE TDR might be \texttt{volume-physics.tex}. 

Please do not edit this file.

%%%%%%%%%%%%%%%%%%%%%%%%%%%%%%%%%%%%%%%%%%%%%%%%%%%%%%%%%%%%%%%%%%%%
\section{The Content Files: Text and Figures}
\label{sec:tech-contentfiles}

These are the content files that you create and/or edit. Depending on the document, they may be 
collected at the top directory level of the repository or under a subdirectory. Images typically go under a \texttt{figures} directory.  

If several volumes of a document are collected into a single repository, each under its own subdirectory, it is recommended to have a \texttt{figures} directory under each subdirectory. Only figures common to multiple volumes would be placed in the top-level \texttt{figures} directory.

%%%%%%%%%%%%%%%%%%%%%%%%%%%%%%%%%%%%%%%%%%%%%%%%%%%%%%%%%%%%%%%%%%%%
\section{The \LaTeX{} Class File}
\label{sec:tech-classfile}

All of the \LaTeX{} configuration for the document that pertains to
the general LBNF/DUNE style is in the
\texttt{td-pdr.cls} file. (The name of this file may change from document to document; it will be the only \texttt{xxx.cls} file.)
The class takes some options that control high-level style, e.g.,:

\begin{verbatim}
\newcommand{\toprowrule}{
  \arrayrulecolor{gray}
  \specialrule{1.2pt}{0pt}{1pt}
  \arrayrulecolor{black}
}
\end{verbatim}

Please do not edit this file.

%%%%%%%%%%%%%%%%%%%%%%%%%%%%%%%%%%%%%%%%%%%%%%%%%%%%%%%%%%%%%%%%%%%%
\section{Document-wide \LaTeX{} Files in \texttt{common/}}
\label{sec:tech-common}

Multi-volume documents are supported and should use the
\texttt{common/} subdirectory to house any information that is to be
shared between volumes.
Because we reuse formatting from one document to the next, the single-volume documents (e.g., the ProtoDUNE-SP TDR) also use this directory.
The expected ``common'' files include:
\begin{description}
\item[\texttt{defs.tex}] macros defining common terms. You may want to work with your team to determine an appropriate set of them, then enter them into this file.
\item[\texttt{units.tex}] macros defining how to use quantities and units. You may want to add some. 
\item[\texttt{preamble.tex}] a file that largely includies the above files along with
  any extra \LaTeX{} configuration that goes before the document
  environment begins.  Please do not edit this file.
\item[\texttt{glossary.tex}] containing words and abbreviations that you want in a glossary.
\item[\texttt{citedb.bib}] the BibTeX database file (name may change; `bib' will remain the same). You will add your references here in the proper format; examples are provided at the head of the file.
\item[\texttt{init.tex}] Any content that goes in the document
  environment but before the first chapter. 
  It typically should set the title page, ToC/LoF/LoT. Please do not edit this file.
\item[\texttt{final.tex}] Any content that goes in the document after
  the last chapter. 
  It typically contains the commands related to producing the glossary and the
  bibliography.  Please do not edit this file.
  \item[\texttt{clean.sh}] When compiled in unix with pdflatex or similar, intermediate files get created. This script removes all the generated files except the output pdf.
\end{description}

The content of many of these files should in part be kept in sync
across all document repositories.

Specific documents may place additional files in this \texttt{common/}
directory that should be shared document-wide.
For example, volumes of the TDR may include a standard \texttt{common/intro.tex}.

%%%%%%%%%%%%%%%%%%%%%%%%%%%%%%%%%%%%%%%%%%%%%%%%%%%%%%%%%%%%%%%%%%%%
\section{Getting the Files and Compiling after your Changes}
\label{sec:tech-compile}

Detailed instructions for getting the files and compiling a volume are
given on main GitHub page for the repository supplying this document at
\url{https://github.com/DUNE/document-guidance}.
  
