\chapter{Technical}

This chapter describes some of the more technical aspects of the CDR.

%%%%%%%%%%%%%%%%%%%%%%%%%%%%%%%%%%%%%%%%%%%%%%%%%%%%%%%%%%%%%%%%%%%%
\section{Getting the Files and Compiling after your Changes}
\label{sec:getfiles-compile}

Detailed instructions for getting the files and compiling a volume are
given on main GitHub page for the repository supplying this document at
\url{https://github.com/DUNE/document-guidance}.

\fixme{Note that any acronyms list that a volume has requires an extra
  step to be included; this step is not documented there.
  The technical editors will take care of it.}

%%%%%%%%%%%%%%%%%%%%%%%%%%%%%%%%%%%%%%%%%%%%%%%%%%%%%%%%%%%%%%%%%%%%
\section{The \LaTeX{} CDR class}

All of the \LaTeX{} configuration for the document that pertains to
the general ``CDR'' style is in the
\texttt{cdr.cls} file.
The class takes some options that control high-level style:

\begin{description}
\item[\texttt{final}] this removes all extra markup useful during
  editing phase.
\end{description}

%%%%%%%%%%%%%%%%%%%%%%%%%%%%%%%%%%%%%%%%%%%%%%%%%%%%%%%%%%%%%%%%%%%%
\section{Volume-generic \LaTeX{} files}

Multi-volume documents are supported and should use the
\texttt{common/} subdirectory to house any information that is to be
shared between volumes.
Single-volume documents will also supply some files here.
The expected ``common'' files include:
\begin{description}
\item[\texttt{defs.tex}] macros defining common terms.
\item[\texttt{units.tex}] macros defining how to use quantities and units.
\item[\texttt{preamble.tex}] largely including the above to files but
  any extra \LaTeX{} configuration that goes before the document
  environment begins.
\item[\texttt{citedb.bib}] the BitTeX database.
\item[\texttt{init.tex}] Any content that goes in the document
  environment but before the first chapter.
  It typically should contain title page, ToC/LoF/LoT.
\item[\texttt{final.tex}] Any content that goes in the document after
  the last chapter.
  It typically should contain the commands related to producing the
  bibliography.
\end{description}

The content of many of these files should in part be kept in sync
across all document repositories.

Specific documents may place additional files in this \texttt{common/}
directory that should be shared among the document's volumes.
For example, all volumes of the CDR include an \texttt{intro.tex}
(which further includes \texttt{supp-doc-list.tex}).

%%%%%%%%%%%%%%%%%%%%%%%%%%%%%%%%%%%%%%%%%%%%%%%%%%%%%%%%%%%%%%%%%%%%
\section{The main file}

A new main file will likely be started only by the technical editors.
To start a new volume, copy the \texttt{guidance.tex} to a new
name and edit as directed by the comments.  

\begin{enumerate}
\item Set the graphics path
\item Redefine the volumes sub-title
\item Input each chapter file.
\end{enumerate}

