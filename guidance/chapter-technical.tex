\chapter{Important files in the GitHub document repository}
\label{ch:tech}

This chapter describes some of the more technical aspects of the LBNF/DUNE \LaTeX setup. Various aspects of the configuration are pulled out into separate files to keep things well organized.  In general, authors should 
not edit the files described in this chapter. Often authors need to include citations, and occasionally may want to add acronyms or definitions. This is ok; just please don't touch any other configuration-related files!

%%%%%%%%%%%%%%%%%%%%%%%%%%%%%%%%%%%%%%%%%%%%%%%%%%%%%%%%%%%%%%%%%%%%
\section{The main file}
\label{sec:tech-mainfile}

A new main file will likely be started only by the technical editors. The main file

\begin{enumerate}
\item sets the document class (see Section~\ref{sec:tech-classfile};
\item sets the graphics path;
\item inputs the preamble;
\item opens the document environment;
\item inputs the ``init'' file;
%\item Redefine the volumes sub-title -- when document is split into volumes; this TDR is not
\item inputs each chapter file;
\item inputs the ``final'' file (which provides the bibliographical information); and
\item closes the document environment.
\end{enumerate}

To start a new document, you would copy the \texttt{guidance.tex} to a new
name and edit as directed by the comments.  As an example, the main file of the physics volume of the DUNE TDR is \texttt{volume-physics.tex}. (Please don't touch it!) 



%%%%%%%%%%%%%%%%%%%%%%%%%%%%%%%%%%%%%%%%%%%%%%%%%%%%%%%%%%%%%%%%%%%%
\section{The \LaTeX{} class file}
\label{sec:tech-classfile}

All of the \LaTeX{} configuration for the document that pertains to
the general LBNF/DUNE style is in the
\texttt{td-pdr.cls} file. (The name of this file may change from document to document; it will be the only \texttt{xxx.cls} file.)
The class takes some options that control high-level style, e.g.,:

\begin{description}
\item[\texttt{final}] this removes all extra markup used during the
  editing phase.
\end{description}


%%%%%%%%%%%%%%%%%%%%%%%%%%%%%%%%%%%%%%%%%%%%%%%%%%%%%%%%%%%%%%%%%%%%
\section{Document-wide \LaTeX{} files in \texttt{common/}}
\label{sec:tech-common}

Multi-volume documents are supported and should use the
\texttt{common/} subdirectory to house any information that is to be
shared between volumes.
Because we reuse formatting from one document to the next, the single-volume documents (e.g., the ProtoDUNE-SP TDR) also use this directory.
The expected ``common'' files include:
\begin{description}
\item[\texttt{defs.tex}] macros defining common terms.
\item[\texttt{units.tex}] macros defining how to use quantities and units.
\item[\texttt{preamble.tex}] a file that largely includies the above files along with
  any extra \LaTeX{} configuration that goes before the document
  environment begins.
\item[\texttt{acronyms.tex}] containing acronyms, abbreviations and so on; there may be separate ones per volume.
\item[\texttt{citedb.bib}] the BibTeX database (name may change; `bib' will remain the same).
\item[\texttt{init.tex}] Any content that goes in the document
  environment but before the first chapter.
  It typically should set the title page, ToC/LoF/LoT.
\item[\texttt{final.tex}] Any content that goes in the document after
  the last chapter.
  It typically contains the commands related to producing the
  bibliography.
\end{description}

The content of many of these files should in part be kept in sync
across all document repositories.

Specific documents may place additional files in this \texttt{common/}
directory that should be shared document-wide.
For example, most (all but intro) volumes of the TDR will include a standard \texttt{common/intro.tex}.

%%%%%%%%%%%%%%%%%%%%%%%%%%%%%%%%%%%%%%%%%%%%%%%%%%%%%%%%%%%%%%%%%%%%
\section{Getting the files and compiling after your changes}
\label{sec:tech-compile}

Detailed instructions for getting the files and compiling a volume are
given on main GitHub page for the repository supplying this document at
\url{https://github.com/DUNE/document-guidance}.
  
