\chapter{Generalities}
\label{ch:generalities}

This volume gives guidance to authors and editors of DUNE and LBNF documents such as the CDR, TDRs, and so on. It collects ``wisdom'' learned 
producing earlier documents, in particular the LBNE CDR and science document, and we will appreciate 
very much if everyone follows it!  It tries to follows its own guidance, so looking at its \LaTeX{} source can 
provide an example.  

%%%%%%%%%%%%%%%%%%%%%%%%%%%%%%%%%%%%%%%%%%%%%%%%%%%%%%%%%%%%%%%%%%%%
\section{Files}
\label{sec:files}

%%%%%%%%%%%%%%%%%%%%%%%%%%%%%%%%%%%%%
\subsection{For the ProtoDUNE-SP TDR (2016)}
\label{ssec:files-pdunesp-tdr}

You will find the \LaTeX{} and figure content for this TDR
arranged under your top ``protodune-tdr'' directory as follows:

\begin{description}
\item[\texttt{protodune-tdr.tex}] the main file and is found in the top-level
  directory. It sets the document class, sets a graphics path (including all directories that may include image files; here it should only include ``figures''), does a couple of other things, then pulls in each of the chapter content files in the right order.
\item[\texttt{figures/}] top-level subdirectory for any static figures.
\item[\texttt{generated/}] subdirectory holding any generated
  figures (see Section~\ref{sec:figures} for info on generated files)
\item[\texttt{somechapter.tex}] file that holds the LaTeX source (content) for one
  chapter of the document. The main file contains a line to input each of these chapter files.
\end{description}

Some general guidance on choosing file names:

\begin{itemize}
\item Editors may elect to further break up content for a chapter into per-section
  files.
  If so, follow a naming convention such as
  ``\texttt{somechapter-somesection.tex}''.
\item Do not include a chapter or section number in any file
  name.
  Their actual numbers will be determined by high-level ordering by
  the editors. 
\item The editors are expected to maintain the top level
  \texttt{protodune-tdr.tex} files.  Authors should please avoid modifying it.
\item Use the \texttt{figures/} subdirectory for static figures (see
  Section~\ref{sec:figures}). For how to include generated figures, see
  Section~\ref{sec:plots}.
\item If you find something inadequate or have questions, consult with either: \\
  Anne Heavey, aheavey@fnal.gov, 630-840-8039 (technical editor)\\
  Brett Viren, bv@bnl.gov (help on graphics, \LaTeX{} and git machinery)
\end{itemize}


%%%%%%%%%%%%%%%%%%%%%%%%%%%%%%%%%%%%%
\subsection{For the CDR (2015)}
\label{ssec:files-cdr}

An entire document may consist of one or more \textit{volumes} (eg,
the CDR consists of several, only some of which are written in
\LaTeX{}.
You will find the \LaTeX{} and figure content for a given volume
arranged like:

\begin{description}
\item[\texttt{NAME.tex}] the main file and is found in the top-level
  directory. It generally has no significant content itself but
  includes content through other chapter or other files.
\item[\texttt{figures/}] top-level subdirectory for any static figures
  shared by more than one volume in the repository.
\item[\texttt{NAME/}] subdirectory holding all content for one particular volume.
\item[\texttt{NAME/chapter-CHAPTERNAME.tex}] holds the content for one
  chapter of a volume.
\item[\texttt{NAME/figures/}] subdirectory holding any static volume-specific figures
\item[\texttt{NAME/generated/}] subdirectory holding any generated
  figures (see Section~\ref{sec:figures} for info on generated files)
\end{description}

Where \texttt{NAME} is some label for the volume (the name for the
volume you are reading now is ``\texttt{guidance}'').
Multi-volume documents may pick some file naming convention (eg, the
CDR uses ``\texttt{volume-VOLUMENAME}''.
Specific files hold chapter-level content an are named after some
short, descriptive label.
The file holding this sentence is ``\texttt{guidance/chapter-general.tex}''.
Some general guidance on choosing these file names:

\begin{itemize}
\item Editors may elect to further break up content into per-section
  files.
  If so, follow a naming convention such as
  ``\texttt{chapter-CHAPTERNAME-section-SECTIONNAME.tex}''.
\item Do not include a volume, chapter or section numbers in any file
  name.
  Their actual numbers will be determined by high-level ordering by
  the editors. 
\item The editors are expected to maintain the top level
  \texttt{NAME.tex} files. General authors should avoid modifying it.
\item Use the \texttt{figures/} subdirectory for static figures (see
  Section~\ref{sec:figures}). For how to include generated figures see
  Section~\ref{sec:plots}.
\end{itemize}

If you find something inadequate or have questions, consult with either: \\
  Anne Heavey, aheavey@fnal.gov, 630-840-8039 (technical editor)\\
  Brett Viren, bv@bnl.gov (help on graphics, \LaTeX{} and git machinery)



%%%%%%%%%%%%%%%%%%%%%%%%%%%%%%%%%%%%%%%%%%%%%%%%%%%%%%%%%%%%%%%%%%%%
\section{Figure Format}
\label{sec:figure-format}

It is essential to use high-quality, efficiently sized figures (aka
``graphics'').
You may be asked to re-make any that egregiously violate some some
basic standards.
These standards are in place to avoid sub-optimal figures, bloated
files sizes, and delayed publishing schedules.  
Often the best thing to do is to \textbf{not} process or attempt to
optimize a graphic but provide the editors access to the most ``raw''
graphic which your software can produce.
If at all uncertain, please contact the technical editors.
The rest of this section provides guidance on how to create optimal
figures.

%%%%%%%%%%%%%%%%%%%%%%%%%%%%%%%%%%
\subsection{Graphic Types}
\label{sec:graphic-types}

There two basic graphic content types; these are important to understand:

\begin{description}
\item[raster] a two dimensional array of pixels
\item[vector] a two dimensional drawing description language
\end{description}

The documents compile with \texttt{pdflatex} and so may use graphics
in PDF, JPEG or PNG file formats.
These formats have specific best uses:

\begin{description}
\item[JPEG] use for photographs
\item[PDF] use of any line drawings, plots, illustrations
\item[PNG] use due to some inability to produce proper JPEG or PDF (contact editors)
\end{description}

It is possible (though unwise) to store inherently raster information
in PDF or to rasterize inherently vector information into JPEG or PNG.
\textbf{This is the main cause for bloated, low-quality graphics.}
Here are some guidelines addressing this common problem:

\begin{itemize}
\item Only save photographic images to JPEG, avoid re-saves.
\item Save line drawings, plots or illustrations directly to vector PDF.
\item Follow special guidance on annotation (see Section~\ref{sec:annotate}).
\item Never convert any raster data (JPEG/PNG) to PDF.
\item Never raster what is really vector data in to a JPEG/PNG.
\item Never use MicroSoft PowerPoint for any figure as it tends to lead to poor quality and bloated files.
\item Do save using native application formats to allow later
  modification
\item Leave any potential format conversions to the technical editors.
\item Consider providing plots as easy-to-reproduce ROOT, Python or
  other scripts.
  (see Section~\ref{sec:plots})
\end{itemize}

\noindent If authors find these guidelines can not be followed for any
given graphic, please contact the technical editors.   

%%%%%%%%%%%%%%%%%%%%%%%%%%%%%%%%%%
\subsection{Plots}
\label{sec:plots}

Where possible, it is recommended that any plots be submitted in a
form that allows easy reproduction from data in the course of building
the document.
This allows the technical editors to attempt to apply consistent
in-plot fonts, colors, wording.
(More info to be added.)

%%%%%%%%%%%%%%%%%%%%%%%%%%%%%%%%%%
\subsection{Annotated Figures}
\label{sec:annotate}

One common figure type is to take a figure and annotate it with
arrows, labels, etc.
Ideally you will do this directly in LaTeX, for example using TikZ.
If you can't do that, then take care not to produce a bloated,
low-quality graphic, and please choose fonts and colors that ``work''
with the document.
If at all possible, provide the file in a format which can be further
edited and which does not turn raster data into PDF nor vector data
into JPEG/PNG.
If this can't be avoided and if the underlying graphic is JPEG then
produce the final version in JPEG and not PNG nor PDF.
If the annotation is on top of an original vector drawing and your
annotation software will retain the vector information, save it as
PDF.

