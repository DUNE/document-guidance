\chapter{General Information}
\label{ch:generalities}

This volume gives guidance to authors and editors of DUNE and LBNF documents such as the CDR, TDRs, and so on. It collects ``wisdom'' learned from 
producing earlier documents, including the DUNE CDR, and we will appreciate 
very much if everyone follows it!  It tries to follows its own guidance so that its \LaTeX{} source
provides an example.  

This has been updated for the DUNE TDR, which will likely have multiple volumes. Earlier versions of this guidance document were updated for other important documents, e.g., the CDR and the ProtoDUNE-SP TDR.  These versions can be found at \url{https://github.com/DUNE/document-guidance/releases/}.


%%%%%%%%%%%%%%%%%%%%%%%%%%%%%%%%%%%%%
\section{What's on the Document Guidance GitHub web page?}
\label{ssec:files-webpage}

Some guidance is on the web page \url{https://github.com/DUNE/document-guidance}:

\begin{itemize}
\item How to get started with GitHub and get the files that you need to edit. See both ``Getting Started'' (at the top) and ``Repository'' further down.  ``Repository'' provides commands to use for interacting with GitHub.
\item How to build the document, in draft format (with extra markup) after you have it downloaded to your local machine.
\item How to build a print-ready document -- even though in principle you don't need to do this, you may want to see how it looks. The technical editors will build the final document.
\item How to access builds of the document run automatically from the content on GitHub.
\item How to get away without authoring in \LaTeX{}...
\end{itemize}

Note about Draft vs. Print format: 

Draft format gives you line numbers, ``fixme'' and other markup notes (see Chapter~\ref{ch:review}), 
faint grey notes (enclosed in a box) near labeled items (headings, figures, tables, etc.) to help you know 
what label was used to reference each particular thing.  
It's easiest to trace things when the labels are meaningful, and for figures, when they match the image filename, so please choose labels carefully. Labels are discussed in Section~\ref{sec:intra-doc-ref}.


%%%%%%%%%%%%%%%%%%%%%%%%%%%%%%%%%%%%%%%%%%%%%%%%%%%%%%%%%%%%%%%%%%%%
\section{Files and filenames}
\label{sec:files}

An entire document may consist of one or more \textit{volumes}; e.g.,
the DUNE/LBNF TDR will likely consist of several, only some of which may be written in
\LaTeX{}.
Within the overall TDR directory, you will find the \LaTeX{} and figure content for all the \LaTeX{} volumes. There are subdirectories for each volume. The content for a given volume is
arranged like:

\begin{description}
\item[\texttt{volname.tex}] the main file and is found in the top-level
  directory. It generally has no significant content itself but
  ``includes'' content contained in other files.
\item[\texttt{figures/}] top-level subdirectory for any static figures
  shared by more than one volume in the repository.
\item[\texttt{volname/}] subdirectory holding all content for one particular volume.
\item[\texttt{volname/chaptername.tex}] holds the content for one
  chapter of a volume.
\item[\texttt{volname/figures/}] subdirectory holding any static volume-specific figures
\item[\texttt{volname/generated/}] subdirectory holding any generated
  figures (see Section~\ref{sec:figures} for info on generated files)
\end{description}

\fixme{fix this after we get the structure set}
Where \texttt{volname} is some label for the volume (the name for the
volume you are reading now is ``\texttt{guidance}'').
Multi-volume documents may pick some file naming convention (e.g., the
TDR may use ``\texttt{volume-VOLUMEname}''.
Specific files hold chapter-level or section-level content and should be named with some
short, descriptive label.
The file holding this sentence is ``\texttt{guidance/chapter-general.tex}''.
Some general guidance on choosing these file names:

\begin{itemize}
\item Editors may elect to further break up content into per-section
  files.
  If so, follow a naming convention such as
  ``\texttt{chaptername-sectionname.tex}''.
\item Do not include a volume, chapter or section numbers in any file
  name.
  Their actual numbers will be determined by high-level ordering by
  the editors. 
\item The editors are expected to maintain the top level
  \texttt{name.tex} files. Other authors should avoid modifying it.
\item Use the \texttt{figures/} subdirectory for static figures (see
  Section~\ref{sec:figures}). For how to include generated figures see
  Section~\ref{sec:plots}.
\item If you find something inadequate or have questions, consult with either: \\
  Anne Heavey, aheavey@fnal.gov, 630-840-8039 (technical editor)\\
  Brett Viren, bv@bnl.gov (help on graphics, \LaTeX{} and git machinery)
\end{itemize}



%%%%%%%%%%%%%%%%%%%%%%%%%%%%%%%%%%%%%%%%%%%%%%%%%%%%%%%%%%%%%%%%%%%%
\section{Figure format}
\label{sec:figure-format}

It is essential to use high-quality, efficiently sized figures (aka
``graphics'').
You may be asked to re-make any that egregiously violate some some
basic standards.
These standards are in place to avoid sub-optimal figures, bloated
files sizes, and delayed publishing schedules.  
Often the best thing to do is to \textbf{not} process or attempt to
optimize a graphic but provide the editors access to the most ``raw''
graphic which your software can produce.
If at all uncertain, please contact the technical editors.
The rest of this section provides guidance on how to create optimal
figures.

%%%%%%%%%%%%%%%%%%%%%%%%%%%%%%%%%%
\subsection{Graphic types}
\label{sec:graphic-types}

There two basic graphic content types; these are important to understand:

\begin{description}
\item[raster] a two dimensional array of pixels
\item[vector] a two dimensional drawing description language
\end{description}

The documents compile with \texttt{pdflatex} and so may use graphics
in PDF, JPEG or PNG file formats.
These formats have specific best uses:

\begin{description}
\item[JPEG] use for photographs
\item[PDF] use of any line drawings, plots, illustrations
\item[PNG] use due to some inability to produce proper JPEG or PDF (contact editors)
\end{description}

It is possible (though unwise) to store inherently raster information
in PDF or to rasterize inherently vector information into JPEG or PNG.
\textbf{This is the main cause for bloated, low-quality graphics.}
Here are some guidelines addressing this common problem:

\begin{itemize}
\item Only save photographic images to JPEG, avoid re-saves.
\item Save line drawings, plots or illustrations directly to vector PDF.
\item Follow special guidance on annotation (see Section~\ref{sec:annotate}).
\item Never convert any raster data (JPEG/PNG) to PDF.
\item Never raster what is really vector data in to a JPEG/PNG.
\item Never use MicroSoft PowerPoint for any figure as it tends to lead to poor quality and bloated files.
\item Do save using native application formats to allow later
  modification
\item Leave any potential format conversions to the technical editors.
\item Consider providing plots as easy-to-reproduce ROOT, Python or
  other scripts.
  (see Section~\ref{sec:plots})
\end{itemize}

\noindent If authors find these guidelines can not be followed for any
given graphic, please contact the technical editors.   

%%%%%%%%%%%%%%%%%%%%%%%%%%%%%%%%%%
\subsection{Plots}
\label{sec:plots}

Where possible, it is recommended that any plots be submitted in a
form that allows easy reproduction from data in the course of building
the document.
This allows the technical editors to attempt to apply consistent
in-plot fonts, colors, wording.
(More info to be added.)

%%%%%%%%%%%%%%%%%%%%%%%%%%%%%%%%%%
\subsection{Annotated figures}
\label{sec:annotate}

One common figure type is to take a figure and annotate it with
arrows, labels, etc.
Ideally you will do this directly in \LaTeX{}, for example using TikZ.
If you can't do that, then take care not to produce a bloated,
low-quality graphic, and please choose fonts and colors that ``work''
with the document.
If at all possible, provide the file in a format which can be further
edited and which does not turn raster data into PDF nor vector data
into JPEG/PNG.
If this can't be avoided and if the underlying graphic is JPEG then
produce the final version in JPEG and not PNG nor PDF.
If the annotation is on top of an original vector drawing and your
annotation software will retain the vector information, save it as
PDF.

%%%%%%%%%%%%%%%%%%%%%%%%%%%%%%%%%%%%%%%%%%%%%%%%%%%%%%%%%%%%%%%%%%%%
\section{Structuring a design document}
\label{sec:design-doc}

To make design documents (e.g., CDR, TDR) easier to read, it's a good idea to follow a similar structure in the presentation of each system, subsystem and component. The structure described here assumes that each major project item (e.g., far detector, near detector...) gets a dedicated volume, and each major system within the item gets a chapter.  Depending on the actual report, this may vary. ``Appropriate level of detail'' may also vary.

The different volumes of the DUNE TDR should have similar introductory chapters in case someone is only reading that volume. 
Use common/intro.tex in the intro chapter, then add in 
The intro chapter should be structured as:
\begin{itemize}
\item (per volume) About this volume.
\end{itemize}

Create an introduction section for each chapter.  For chapters about detector systems, use the APA template as a model.
\fixme{I need to link it to here}

\begin{itemize}
\item Begin with: ``The scope of the (name of system) includes the design, procurement, fabrication, testing, delivery and installation (or whatever it includes) of all the subsystems (or components) that comprise it:'' (follow with a list)
\item Include an image of the overall system, labeling its parts. Show how the system fits into the overall L2 project item.
\item Followed this by a list of (major) requirements that the system must satisfy.
\item Check this against the requirements documentation.
\item Describe the overall system to appropriate level of detail.
\end{itemize}

Create a section for each subsystem of the system.

\begin{itemize}
\item Begin with a description of the overall subsystem.
\item Include a labeled image of the subsystem.  
\item List all the components that comprise the subsystem.
\item Start a subsection for each component that describes and illustrates it, to appropriate level of detail.
\item The text in the section should clearly indicate how the subsystem/component satisfies the related requirements listed in the intro section.
\item By the end of the chapter, for every requirement listed in the intro section, there should exist an explanation of how it will be satisfied.
\end{itemize}
