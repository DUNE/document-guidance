\chapter{General Information}
\label{ch:gen}

%%%%%%%%%%%%%%%%%%%%%%%%%%%%%%%%%%%%%%%%
\section{Introduction}

DUNE has developed standards and conventions for large documents, design reports in particular, that contributors are asked to follow. An infrastructure that reflects these standards is implemented in LaTeX on GitHub for each document. This guidance document provides instructions for using the infrastructure and guidelines for contributing high-quality text. It is written according to these conventions so that its \LaTeX{} source can be used as an example.  

This has been updated for the DUNE TDR. Earlier versions of this guidance document were updated for other important documents, e.g., the CDR and the ProtoDUNE-SP TDR.  These versions can be found on GitHub at \url{https://github.com/DUNE/document-guidance/releases/}.

This chapter provides a quick summary of \textit{some} of the important LaTeX conventions described in this document. More information is found in the other chapters of this document.

%%%%%%%%%%%%%%%%%%%%%%%%%%%%%%%%%%%%%%%%
\section{Definitions}
We have two ways of enforcing consistency of terminology and LaTeX rendering: definitions, which define simple LaTeX code to substitute for complicated or error-prone code, and a glossary database. Both allow definition of a macro to be used in place of the actual term. A separate files exists for unit definitions. In addition, the glossary provides a definition of the given term or abbreviation. 

Please use both of these features as appropriate, and feel free to add new ones to the appropriate files.   
%%%%%%%%%%%%%%%%
\subsection{Simple Definitions}
\begin{framed}
Find the full description in Sections~\ref{sec:latex-terms} and \ref{sec:num-units}.

We recommend that you use (and add to, as needed) macros defined in \texttt{common/defs.txt} for complicated expressions like 
\begin{verbatim}
$\bar\nu_\mu$.
\end{verbatim}
 Clearly, typing a macro like 
 \begin{verbatim}
\anumu
\end{verbatim} 
is easier and much less error-prone than the full expression.
\end{framed}

%%%%%%%%%%%%%%%%
\subsection{Glossary Terms}
\begin{framed}
Find the full description in Section~\ref{sec:english-terminology}.

Please see the file \verb|common/glossary.tex| for defined terms. To use terms defined there, you will typically use \verb|\dword{label}| in your text.  Variations on this are described in Section~\ref{sec:english-terminology}.
\end{framed}

%%%%%%%%%%%%%%%%
\section{Requirements Tables}
\begin{framed}
Find the full description in Section~\ref{sec:req-tables}.

Requirements (specifications) will be managed in tables. The TDR team will provide a template and guidance.  Please follow the instructions in 
Section~\ref{sec:req-tables} carefully. 
\end{framed}

%%%%%%%%%%%%%%%%
\section{Figures}
\begin{framed}
Find the full description in Section~\ref{sec:latex-figures}.
\begin{verbatim}
    \begin{dunefigure}[optional caption for LoF]{fig:required-label}
     {required full caption (Credit: xyz)}
    \includegraphics[width=0.8\textwidth]{image-filename}
    \end{dunefigure}
\end{verbatim}
To reference: \verb|See Figure~\ref{fig:required-label}|.
Please make \verb|required-label| the same as \verb|image-filename|. 

Please check that your figures are reasonably sized. We may want to add this document to the arXiv, which has a 10\,MB limit per document; bloated figures can use that up quickly. 

Mac users: Please avoid the use of capital letters in graphics filenames; it causes errors that you don't catch when compiling locally on your system. The Mac file systems are not truly case-sensitive, and a mismatch in the capitalization of graphics file names causes problems down the line in the repository.
\end{framed}

%%%%%%%%%%%%%%%%
\section{Tables}
\begin{framed}
Find the full description in Section~\ref{sec:latex-tables}.
\begin{verbatim}
\begin{dunetable}
[The LoT caption]
{cc}
{tab:table-label}
{The full caption that appears above the table.}
  Rows & Counts \\ \toprowrule
  Row 1 & First \\ \colhline
  Row 2 & Second \\ \colhline
  Row 3 & Third \\ 
\end{dunetable}
\end{verbatim}
To reference: \verb|See Table~\ref{tab:table-label}|. The ``cc'' means: ``two columns, both centered.'' You can substitute ``l'' or ``r'' to left- or right-justify a column's contents. Use \verb|p{'width'}| if you need a column's contents to wrap to multiple lines.
\end{framed}

%%%%%%%%%%%%%%%%
\section{Citations}
\begin{framed}.
Find the full description in Section~\ref{sec:latex-cit}.

Citations go in \texttt{common/tdr-citedb.bib}. When adding citations to this file, copy the BibTeX rendering of the citation from \url{http://inspirehep.net} if it exists there. (We have left a well-populated \texttt{common/citedb.bib} in the TDR directory in case you want to search for a citation
 that may have appeared in the CDR, or other DUNE/LBNF document.) E.g.,
\begin{verbatim}
 @article{Gando:2010aa,
      author         = {A.~Gando and others},
      title          = "{Constraints on $\theta_{13}$ from A Three-Flavor ...}",
      journal        = {Phys.Rev.D},
      volume         = {83},
      pages          = {052002},
      year           = {2011},
      note           = {arXiv:1009.4771 [hep-ex]},
}
\end{verbatim}
 To reference it within a file, use, e.g.,  \verb|\cite{Gando:2010aa}|.
\end{framed}

%%%%%%%%%%%%%%%%
\section{Fixmes}
More info and other options are described in Chapter~\ref{ch:review}.

\begin{framed}
``Fixme'' as in ``fix me!'' If you're not sure about something or you find an error that somebody else has to fix, use:
\begin{verbatim}
\fixme{This is a fixme.}
\end{verbatim}
\end{framed}

%%%%%%%%%%%%%%%%
\section{Sectioning}

\begin{framed}
Find the full description in Section~\ref{sec:latex-sectioning}.
\begin{verbatim}
\chapter{A Chapter}
\label{ch:blah}

%%%%%%%%%%%%%%%%%%%%%%%%%%%%%%%%%%%%%%%%%%%%%%%%%%%%%%%%%%%%%%%%%%%%
\section{A Section}
\label{sec:blah-sec1}

%%%%%%%%%%%%%%%%%%%%%%%%%%%%%%%%%%%%%
\subsection{A Subsection}
\label{sec:blah-sec1-subsec1}

%%%%%%%%%%%%%%%%%%%%
\subsubsection{A Subsubsection}
\label{sec:blah-sec1-subsec1-subsubsec1}
\end{verbatim}
\end{framed}
