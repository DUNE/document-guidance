\chapter{General information}
\label{ch:gen}

This volume gives guidance to authors and editors of DUNE and LBNF documents such as the CDR, TDRs, and so on. It collects ``wisdom'' learned from 
producing earlier documents, including the DUNE CDR, and we will appreciate 
very much if everyone follows it!  It tries to follows its own guidance so that its \LaTeX{} source
provides an example.  

This has been updated for the DUNE TDR, which will likely have multiple volumes. Earlier versions of this guidance document were updated for other important documents, e.g., the CDR and the ProtoDUNE-SP TDR.  These versions can be found at \url{https://github.com/DUNE/document-guidance/releases/}.


%%%%%%%%%%%%%%%%%%%%%%%%%%%%%%%%%%%%%
\section{What's on the Document Guidance GitHub web page?}
\label{sec:gen-webpage}

Some guidance is on the web page \url{https://github.com/DUNE/document-guidance}:

\begin{itemize}
\item How to get started with GitHub and get the files that you need to edit. See both ``Getting Started'' (at the top) and ``Repository'' further down.  ``Repository'' provides commands to use for interacting with GitHub.
\item How to build the document, in draft format (with extra markup) after you have it downloaded to your local machine.
\item How to build a print-ready document -- even though in principle you don't need to do this, you may want to see how it looks. The technical editors will build the final document.
\item How to access builds of the document run automatically from the content on GitHub.
\item How to get away without authoring in \LaTeX{}...
\end{itemize}

Note about Draft vs. Print format: 

Draft format gives you line numbers, ``fixme'' and other markup notes (see Chapter~\ref{ch:review}), 
faint grey notes (enclosed in a box) near labeled items (headings, figures, tables, etc.) to help you know 
what label was used to reference each particular thing.  
It's easiest to trace things when the labels are meaningful, and for figures, when they match the image filename, so please choose labels carefully. Labels are discussed in Section~\ref{sec:latex-intra-doc-ref}.

%%%%%%%%%%%%%%%%%%%%%%%%%%%%%%%%%%%%%%%%%%%%%%%%%%%%%%%%%%%%%%%%%%%%
\section{What's in the Document Guidance repository?}
\label{sec:gen-repo}


A document pertaining to the overall LBNF and/or DUNE enterprise may consist of multiple \textit{volumes}\footnote{Some volumes may be written in MS Word, external to a GitHub repository.}.  The repository holds all the files needed to compile each volume of the document (or the document itself, if it's not broken into volumes), both content files and additional files that set up the configuration, 
contain auxiliary text (e.g., acronym lists, citations), or provide definitions or macros (defs.txt), etc.  Most of these other files are maintained by the technical editor and you should not modify them. You may of course add citations, acronyms, and so forth.  More information on these files is given in Chapter~\ref{ch:tech}.


The repository is set up with a subdirectory for each volume -- this is where you will do the bulk of your work. In the following, \texttt{volname} represents a label for the volume. The content for a given volume is
arranged as follows:

\begin{description}
\item[\texttt{volname.tex}] the main file (see Section~\ref{sec:tech-mainfile}) for a volume; there will likely be a few of these. They are found in the top-level
  directory. They are have no significant content themselves; they just use
  ``includes'' to pull together content contained in their component chapter files. Please do NOT modify any volume file!
\item[\texttt{figures/}] top-level subdirectory for any static figures
  shared by more than one volume in the repository. Typically your figures will be volume-specific and you won't add them here.
\item[\texttt{volname/}] subdirectory holding all content for one particular volume. You will work within a given volume subdirectory.
\item[\texttt{volname/chaptername.tex}] holds the content for one chapter of a volume. You will add or edit content in one or more chapter files.
\item[\texttt{volname/figures/}] subdirectory holding any static volume-specific figures. Your figures will usually go here.
\item[\texttt{volname/generated/}] subdirectory holding any generated
  figures (see Section~\ref{sec:graphic-plots} for info on generated files). Your ROOT plots, for example, will go here.
\end{description}


Multi-volume documents will have an established file naming convention (e.g., the
TDR may use ``\texttt{volume-volumename}''.
Specific files hold chapter-level or section-level content and should be named with some
short, descriptive label, e.g., ``\texttt{chapter-chaptername}.''

Please note:
\begin{itemize}
\item Editors may elect to further break up content into per-section files. If so, follow a naming convention such as ``\texttt{chaptername-sectionname.tex}''.
\item Please do not include a volume, chapter, section or figure number in any file name. Their actual numbers will be determined at compile time according to the configuration and the volume's main file. 
\item If you have questions, consult with either: \\
  Anne Heavey, aheavey@fnal.gov, 630-840-8039 (technical editor)\\
  Brett Viren, bv@bnl.gov (help on graphics, \LaTeX{} and git machinery)
\end{itemize}

%%%%%%%%%%%%%%%%%%%%%%%%%%%%%%%%%%%%%%%%%%%%%%%%%%%%%%%%%%%%%%%%%%%%
\section{What's in this guidance document?}
\label{sec:gen-doc}

As you might imagine, this document contains GUIDANCE for writing your documents. Guidance on \LaTeX{}, language, abbreviations, figures, and other things. Please read through it so that you understand the LBNF/DUNE style and structure for documents.  There is \LaTeX{} code that you can copy straight from the source for this document and edit to fit your content.  This is supposed to make it easy!


