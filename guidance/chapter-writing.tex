
\chapter{\LaTeX{} Conventions for LBNF/DUNE Documents}
\label{ch:latex-stds}

This chapter provides guidelines on sectioning, inserting figures, tables and references, and formatting numbers and units according to the \LaTeX{} configuration we have set up for LBNF/DUNE documents.

Prepare your section(s) to the level of peer-reviewed publications for an international science journal.
 
Target your introductory section(s) at both scientists from the broad HEP community and at knowledgeable, but not necessarily expert, members of national and international science policy organizations. Recommendation: write the chapter first, then go back and write the introduction to it. It's more likely to be coherent!

Target the body of your section(s) at scientists and engineers from the broader HEP community.


%%%%%%%%%%%%%%%%%%%%%%%%%%%%%%%%%%%%%%%%%%%%%%%%%%%%%%%%%%%%%%%%%%%%
\section{Defined Terms: Macros and the Glossary}
\label{sec:latex-terms}

To enforce consistency among frequently used terms, we have defined \LaTeX{} macros for several names, expressions and parameter values in the file \texttt{common/defs.tex}.  We also use a system called ``DUNE Words'' to enforce consistency of terminology. (Thank you, Brett.)  Terms are defined in the file \texttt{common/glossary.tex}.

%%%%%%%%%%%%%%%%%%%%%%%%%%%%%%%%%%%%
\subsection{The common/defs.tex File}

You may add terms
to this file; this is useful if the name or term is subject
to multiple ``spellings.''  Please add macros carefully, and test them before you commit. 

For example,
\begin{itemize}
\item \anue is written as \verb|\anue|,
\item \dm{21} is written as \verb|\dm{21}|,
\item \sinst{13} is written as \verb|\sinst{13}|,
\item \numutonumu is written as \verb|\numutonumu|,
\item \ptoknubar is written as \verb|\ptoknubar|,
\item the drift velocity parameter value \driftvelocity is written as \verb|\driftvelocity|,
\item SURF is written as \verb|\surf|, and
\item \minerva is written as \verb|\minerva|.
\end{itemize}

%%%%%%%%%%%%%%%%%%%%%%%%%%%%%%%%%%%%
\subsection{The common/glossary.tex file}

``If you call it a spade here, call it a spade there.'' We want to enforce consistency in terminology. We do this via a glossary filled with ``DUNE Words.''  This means you should check the \verb|common/glossary.tex| for defined terms and use \verb|\dword{label}| for these terms in your text.

 More information, including variations on \verb|\dword{label}|, is given in Section~\ref{sec:english-terminology}, and the details are explained in Brett's (short) document ``DUNE Words'' at \\
 \url{https://dune.bnl.gov/docs/technical-proposal/dune-words.pdf}.



%%%%%%%%%%%%%%%%%%%%%%%%%%%%%%%%%%%%%%%%%%%%%%%%%%%%%%%%%%%%%%%%%%%%
\section{Numbers and Units}
\label{sec:num-units}

\textbf{All} numerical quantities expressed as literal number
\textbf{must} have units unless they are inherently unitless.
The international audience demands consistent use of metric units throughout.  Please follow this standard (fully documented at \url{http://mirrors.sorengard.com/ctan/macros/latex/contrib/siunitx/siunitx.pdf}). 

If and only if  a detector component has been purchased or specified using a US/Imperial or another practical non-standard metric unit, then that unit should be listed as well in parentheses that follow (use of the ambiguous but common \si{\kt} is allowed). For example:

``Holes are drilled to a diameter of \SI{0.635}{cm} (\SI{0.25}{in})... The \SI{0.635}{cm} diameter holes are cleaned with...''

In order to enforce consistency the \texttt{siunitx} package is used
and a collection of common units are defined in
\texttt{common/units.tex}.


%%%%%%%%%%%%%%%%%%%%%%%%%%%%%%%%%%%%
\subsection{Bare Numbers}

To enforce consistent writing of numbers please encase them in the
\verb|\num{}| command:

\begin{itemize}
\item ``\num{100}'' is written as \verb|\num{100}|.
\item ``\num{1000}'' is written as \verb|\num{1000}|.
\item ``\num{123.456}'' is written as \verb|\num{123.456}|.
\item ``\num{1+-2i}'' is written as \verb|\num{1+-2i}|.
\item ``\num{3e45}'' is written as \verb|\num{3e45}|.
\item ``\num{.3e45}'' is written as \verb|\num{.3e45}| (keeps the decimal point before the 3).
\item ``\numlist{10;20;30}'' is written as\verb|\numlist{10;20;30}|.
\end{itemize}

%%%%%%%%%%%%%%%%%%%%%%%%%%%%%%%%%%%%
\subsection{Bare Units}

If you need to write a bare unit, one with not associated number, use \verb|\si{}| (lower case ``si''). Some combinations are
defined in the file \texttt{common/defs.tex}.


\begin{itemize}
\item ``\si{\meter}'' is written \verb|\si{\meter}| (The European spelling \verb|\si{\metre}| also works).
\item ``\si{\square\volt\cubic\lumen\per\farad}'' is written \verb|\si{\square\volt\cubic\lumen\per\farad}|.
\item  ``\si{\volt}'' is written \verb|\si{\volt}|.
\item  ``\si{kg.m.s^{-1}}'' is written either as \verb|\si{kg.m.s^{-1}}| or  \verb|\si{\kilogram\meter\per\second}|.
\item  ``\si{kg.m.s^{-1}}'' is written as \verb|\si[per-mode=symbol]{\kilogram\meter\per\second}|. 
\end{itemize}

%%%%%%%%%%%%%%%%%%%%%%%%%%%%%%%%%%%%
\subsection{Numbers with Units}

When a quantity has a unit, write both the numerical part and the unit
using the \verb|\SI{}{}| command like:

\begin{itemize}
\item ``\SI{120}{\GeV}'' is written as \verb|\SI{120}{\GeV}|,
\item ``\SI{4850}{\ft}'' is written as \verb|\SI{4850}{\ft}|,
\end{itemize}

These are defined in the file \texttt{common/units.tex}.

No hyphens shall be used between numbers and units, even if the number-unit is used as an adjective, e.g., ``a 40 kiloton detector.''  (This is a change from past DUNE documents.) 

%%%%%%%%%%%%%%%%%%%%%%%%%%%%%%%%%%%%
\subsection{Common Compound Units}

There are some common units that rather long to type out each time
especially when we require nice formatting. Again, see \texttt{common/units.tex}.

\begin{itemize}
\item ``per \msr'' is written as \verb|per \msr|, and
\item ``exposure in \ktmwyr{}s'' is written as \verb|exposure in \ktmwyr{}s|.
\end{itemize}


%%%%%%%%%%%%%%%%%%%%%%%%%%%%%%%%%%%%%%%%%%%%%%%%%%%%%%%%%%%%%%%%%%%%
\section{Including ``dunefigures''}
\label{sec:latex-figures}

Instead of using the usual \texttt{figure} environment, please use the custom \texttt{dunefigure}
environment in order to provide for a consistent presentation.
The environment is called with one optional and two required
arguments:

\begin{enumerate}
\item An initial, optional short caption to appear in the List Of Figures (LoF), in square brackets. This caption only needs to
identify the figure uniquely, it does not need to describe it fully.
\item A (required) label for referencing. No spaces are allowed in the label. Curly brackets. Start the label with \texttt{fig:} then use the filename of the figure as the rest of the label, (\texttt{image-filename}, below). 
\item The (required) full caption. Curly brackets, again.
\end{enumerate}

Usually the figure contains a graphic via an \texttt{includegraphics} statement.
The filename is assumed relative to a \texttt{graphicspath} as
mentioned in Section~\ref{sec:tech-mainfile} and as such, please (a) put the file in the right place and (b) don't
 specify any directory parts in its name. 
The file's extension may be omitted. Provide an image credit if appropriate. The entry looks like this (make the width/size appropriate for the image):

\begin{verbatim}
    \begin{dunefigure}[optional caption for LoF]{fig:figure-label}
     {required full caption (Credit: xyz)}
    \includegraphics[width=0.8\textwidth]{image-filename}
    \end{dunefigure}
\end{verbatim}

Do not prepend any directory information to the image-filename in the includegraphics line; it will cause an error. The \LaTeX setup that we have already takes care of finding the image. 

Mac users: Please avoid the use of capital letters in graphics filenames; it causes errors that you don't catch when compiling locally on your system. The Mac file systems are not truly case-sensitive, and a mismatch in the capitalization of graphics file names causes problems down the line in the repository.


Remember, ``image-filename'' should match ``figure-label.'' This makes it MUCH easier to find associated references and figures in the \LaTeX{}  source.  Also, we recommend starting the filename with an abbreviated form of the subsystem it relates to, e.g., for the APAs,

\begin{itemize}
\item image filename = apa-frame-and-wires
\item label = fig:apa-frame-and-wires
\item reference = fig:ig:apa-frame-and-wires
\end{itemize}

Example:
\begin{dunefigure}[An aerial photograph of Fermilab]{fig:fermilab-aerial}{An aerial photograph of Fermilab
    showing Wilson Hall and surrounding accelerator rings (Photo: Fermilab
    Visual Media Services).}
  \includegraphics[width=0.8\textwidth]{fermilab-aerial}
\end{dunefigure}


An example can be seen in Figure~\ref{fig:fermilab-aerial}, which is created
with the \LaTeX{} shown in Figure~\ref{fig:aerial-latex}.  Notice that in the reference to the figures, we need to prepend \texttt{fig:} to the label. (See Section~\ref{sec:latex-intra-doc-ref}.)

\begin{dunefigure}[\LaTeX{} source for figure inclusions.]{fig:aerial-latex}{\LaTeX{} source showing how to include Figure~\ref{fig:fermilab-aerial}.}
\begin{verbatim}
    \begin{dunefigure}[An aerial photograph of Fermilab]{fig:fermilab-aerial}
       {An aerial photograph of Fermilab showing Wilson Hall and 
       surrounding accelerator rings (Photo: Fermilab Visual Media Services).}
    \includegraphics[width=0.8\textwidth]{fermilab-aerial}
    \end{dunefigure}
\end{verbatim}
\end{dunefigure}

See Chapter~\ref{ch:graphics} for guidelines on the graphics files themselves.

\FloatBarrier

%%%%%%%%%%%%%%%%%%%%%%%%%%%%%%%%%%%%%%%%%%%%%%%%%%%%%%%%%%%%%%%%%%%%
\section{Including ``dunetables''}
\label{sec:latex-tables}

Like figures, we use a special environment, \texttt{dunetable} for
tables to achieve a degree of consistency.
This replaces the usual double \texttt{table} + \texttt{tabular} environments.
The \texttt{dunetable} environment takes one optional and three
required arguments:

\begin{enumerate}
\item An initial, optional short caption for the List of Tables (LoT). Square brackets.
\item The tabular column specification. (Curly brackets for the last three items.)
\item A label for referencing. Start the label with \texttt{tab:}. 
\item The full caption.
\end{enumerate}

Inside the actual contents of the table you are required to provide an
initial row containing the headings for the table's rows followed by a
\texttt{toprowrule} macro.
Following every regular row (except the last) you should include a
\texttt{colline} macro.
Both of these take the place of the usual \texttt{hline}.

\begin{dunetable}
[The LoT caption]
{cc}
{tab:table-label}
{This is a sample table.}
  Rows & Counts \\ \toprowrule
  Row 1 & First \\ \colhline
  Row 2 & Second \\ \colhline
  Row 3 & Third \\ 
\end{dunetable}

\noindent Table~\ref{tab:table-label} is thus made like (arguments can span lines):

\begin{verbatim}
\begin{dunetable}
[The LoT caption]
{cc}
{tab:table-label}
{This is a sample table.}
  Rows & Counts  \\ \toprowrule
  Row 1 & First  \\ \colhline
  Row 2 & Second \\ \colhline
  Row 3 & Third  \\ 
\end{dunetable}

See Table~\ref{tab:table-label}...
\end{verbatim}

The ``cc'' means: ``two columns, both centered.'' You can substitute ``l'' or ``r'' to left- or right-justify a column's contents. Use \verb|p{'width'}| (e.g.,\verb|p{0.2\textwidth}| or \verb|p{5cm}| if you need a column's contents to wrap to multiple lines.

Table~\ref{tab:pdecay} shows a more complex example.
See the source for how it is written.
Note that special column specifications are used.

\begin{dunetable}
[Efficiencies and background rates for nucleon decay modes]
{$L^c^c^c^c}
{tab:pdecay}
{Efficiencies and background rates for nucleon decay channels of interest for a large underground LArTPC, and 
comparison with water Cherenkov detector capabilities}
Decay Mode & \multicolumn{2}{^c}{Water Cherenkov} & \multicolumn{2}{^c}{Liquid Argon TPC} \\
\rowtitlestyle              % ``\rowstyletitle'' is needed here to bold the 2nd row of header text.
& Efficiency & Background & Efficiency & Background \\ \toprowrule
$p \rightarrow K^+ \overline{\nu}$ & 19\% & 4 & 97\% & 1 \\ \colhline
$p \rightarrow K^0 \mu^+$ & 10\% & 8 & 47\% & $<2 $ \\ \colhline
$p \rightarrow K^+ \mu^- \pi^+$ & & & 97\% & 1 \\ \colhline
$n \rightarrow K^+ e^- $ & 10\% & 3 & 96\% & $<2$ \\ \colhline
$n \rightarrow e^+\pi^-$ & 19\% & 2 & 44\% & 0.8 \\
\end{dunetable}

\FloatBarrier

%%%%%%%%%%%%%%%%%%%%%%%%%%%%%%%%%%%%%%%%%%%%%%%%%%%%%%%%%%%%%%%%%%%%
\section{Tables Specifically for the DUNE TDR}
\label{sec:tables}

In this section we show a sample table (shown for SP HV) for each standardized table to be used in detector-system chapters of the TDR. 

%%%%%%%%%%%%%%%%%%%%%%%%%%%%%%%%%%%%%%
\subsection{Requirements/Specifications Tables}
\label{sec:tables-req}

Enumeration of requirements is an important addition to the TDR, and a systematic way of doing this is in place.  

In DUNE DocDB 11074~\cite{bib:docdb11074} you will find a template Excel file in which to enter your requirements or specifications.  So far this includes only SP requirements. The file has several sheets (tabs). The TDR team has an automated way of turning the data in this file into LaTeX, hence the detailed instructions that follow:

\begin{enumerate}
\item Download the file from DocDB 11074~\cite{bib:docdb11074}.
\item Read through the top-level requirements in the sheet labeled ``Top-level to review.''  The top five (colored yellow) are considered global; they may appear in all subsystem chapters (TBD). Of the others (orange and blue), some will apply to your chapter and others will not.
\item Enter the unique ID numbers (column A) of the global requirements that apply to your chapter into column A of the tab labeled ``Your selected top level reqs.''
\item In ``Your chapter'' sheet, note the guidelines given in rows 1 and 2. Also note that a couple of example rows are given.
\item  Enter any additional requirements or specifications for your chapter into this sheet using proper LaTeX.  See Figure~\ref{fig:spec-fields}.
\item The TDR will pick up the brief rationale and validation fields, not the full ones. Please fill out both full and brief; this information may be used in other contexts in the future.
\item In the validation field, be sure to include ProtoDUNE impacts and simulation studies, as appropriate. You may include additional validation information, as well.  
\item Please be concise!  We expect no more than five or ten additional specifications here. Questions on content? Ask Tim and Sam.
\item Delete the example rows.
\item Save your spreadsheet as \textit{DUNE-req-spec-(your chapter code)-DDMMMYYYY.xlsx}, e.g., DUNE-req-spec-SP-PDS-30oct2018.xlsx.
\item Upload your completed spreadsheet to your existing requirements DocDB entry, if you have one. (It should be labeled with the topic ``Requirements'' to allow easy searching.) If you don't, create a new DocDB entry for it. 
\item Whenever you update your spreadsheet, remember to upload it to DocDB into a new version. 
\item Be aware that if any of the top-level requirements change, you will need to review and possibly update your spreadsheet.  The TDR team will provide the information on what's been updated.
\end{enumerate}

\begin{dunefigure}[Guidelines for specifications]{fig:spec-fields}
       {Guidelines for specifications}
    \includegraphics[width=0.8\textwidth]{spec-fields}
    \end{dunefigure}
The TDR team will turn your Excel files into LaTeX tables and the generated files will reside under the ``generated'' folder in the repository structure. The TDR team will work with you to insert them into your chapter files.

%%%%%%%%%%%%%%%%%%%%%%%%%%%%%%%
\subsection{Interface Document Tables}
\label{sec:tables-intfc}

We have a standard table format for linking to interface documents in the DUNE docdb. In the first column, put the system to which you are interfacing; use the glossary entry for the system, if it exists. For the second column, we provide a macro to use to add linked references,  \verb|\citedocdb{}|. The parameter is the interface document's docdb number. 

The output of the macro is a hyperlink to the version of the document in the docdb followed by a citation.  The macro requires that the citation's label be of the form: \verb|bib:docdbNNNN|, where NNNN is the docdb number, and that the URL be of the form \\ \verb|https://docs.dunescience.org/cgi-bin/private/ShowDocument?docid=NNNN&asof=2019-7-15|.



\begin{verbatim}
\begin{dunetable}
[High Voltage System Interface Links]
{p{0.25\textwidth}p{0.5\textwidth}l}
{tab:HVinterfaces}
{High Voltage System Interface Links }   
Interfacing System & Description & Linked Reference \\ \toprowrule
\dword{cisc} & \dword{hv} vs. \dword{lar} level interlock, sensor locations in high-field regions, cold/warm camera coverage, \dword{hv} signal monitoring, etc. & \citedocdb{6787}  \\ \colhline
... & & \citedocdb{nnnn}                   \\ \colhline
(last item)& & \citedocdb{}                \\
\end{dunetable}
\end{verbatim}


This is a reference to Table~\ref{tab:HVinterfaces} (\verb|\ref{tab:HVinterfaces}|).
%\fixme{Finalize with Brett-24oct}
\begin{dunetable}
{p{0.25\textwidth}p{0.5\textwidth}l}
{tab:HVinterfaces}
{High Voltage System Interface Links }   
Interfacing System & Description & Linked Reference \\ \toprowrule

\dword{cisc} & \dword{hv} vs. \dword{lar} level interlock, sensor locations in high-field regions, cold/warm camera coverage, \dword{hv} signal monitoring, etc. & \citedocdb{6787} \\ \colhline
 ... & & \citedocdb{nnnn}                   \\ \colhline
(last item)& & \citedocdb{}                 \\
\end{dunetable}

%%%%%%%%%%%%%%%%%%%%%%%%%%%%%%%
\subsection{Risk Tables}
\label{sec:tables-risk}

As of March 2019, risk tables will be autogenerated similarly to specifications tables.  The Excel format (not quite final as of 21 March) is in DocDB (??). The required fields are:

\begin{itemize}
\item Risk ID (format RT or RO (for risk threat or opportunity), followed by consortium abbreviation (e.g., SP-APA) followed by a sequential number (001, 002, etc.) E.g., RT-SP-APA-002);
\item  Risk title;
\item  Latex label (for processing), this should identify risk and have no spaces, e.g., apa-prod-quality;
\item  Description (can be longer, will not appear in table but may be used later);
%\item  Consortium (e.g., SP APA);
\item    Risk Mitigation (full, may go in accompanying text, TBD)
\item    Risk Mitigation (brief to go in table)
\item    Probability (HML, post-mitigation, where: High >25\%, Med 10-20\%, Low <10\%)
\item   Cost impact (HML, post-mitigation, where: High >20\% increase, Med 5-20\% inc, Low <5\% inc);
\item   Schedule impact (HML, post-mitigation, High >6 mo delay, Med 2-6 mo delay, Low <2 mo delay);
\end{itemize}

A sample might look like: 
\begin{dunetable}
[Risk Table]
{p{0.1\textwidth}p{0.2\textwidth}p{0.3\textwidth}p{0.1\textwidth}p{0.1\textwidth}p{0.1\textwidth}}
{tab:HVrisks}
{My subsystem's post-mitigation risk summary}   
ID & Risk & Mitigation & Probability & Cost Impact & Schedule Impact                \\ \toprowrule
(id 1) & Sapien eget mi proin & Lorem ipsum dolor sit amet, consectetur. & L& M&  L     \\ \colhline
(id 2) & Libero enim sed    &Urna cursus eget nunc scelerisque viverra mauris. & M& L&  M      \\ \colhline
(last id)& risk text   &&&&   \\
\end{dunetable}

\begin{comment}
This is old, pre March 20, 2019
\begin{verbatim}
\begin{dunetable}
[High Voltage System Risk Summary]
{p{0.15\textwidth}p{0.75\textwidth}}
{tab:HVrisks}
{High Voltage System Risk Summary}   
ID & Risk                 \\ \toprowrule
(id 1) & risk text        \\ \colhline
(id 2) & risk text        \\ \colhline
... & ...                 \\ \colhline
(last id)& risk text      \\
\end{dunetable}
\end{verbatim}

This is a reference to Table~\ref{tab:HVinterfaces} (\verb|\ref{tab:HVrisks}|).

\begin{dunetable}
[High Voltage System Risk Summary]
{p{0.15\textwidth}p{0.75\textwidth}}
{tab:HVrisks}
{High Voltage System Risk Summary}   
ID & Risk                   \\ \toprowrule
(id 1) & risk text        \\ \colhline
(id 2) & risk text       \\ \colhline
... & ...                        \\ \colhline
(last id)& risk text     \\
\end{dunetable}
\end{comment}
%%%%%%%%%%%%%%%%%%%%%%%%%%%%%%%
\subsection{Cost Tables}
\label{sec:tables-cost}

\begin{verbatim}
\begin{dunetable}
[Cost Summary]
{p{0.5\textwidth}p{0.2\textwidth}p{0.2\textwidth}}
{tab:XXcostsumm}
{Cost Summary}   
Cost Item & M\&S (k\$ US) & Labor Hours \\ \toprowrule
\rowcolor{dunepeach} Design, Engineering and R\&D &  &     \\ \colhline
 (E.g., Photosensors design) &     &             \\ \colhline
 (E.g., Mechanics design) &     &             \\ \colhline
 &     &             \\ \colhline
 &     &             \\ \colhline
 &     &             \\ \colhline
 &     &             \\ \colhline
\rowcolor{dunepeach} Production Setup &  &     \\ \colhline
 (E.g., Photosensors production setup)  &     &             \\ \colhline
 &     &             \\ \colhline 
 &     &             \\ \colhline
 &     &             \\ \colhline 
 &     &             \\ \colhline
 &     &             \\ \colhline
\rowcolor{dunepeach} Production &  &     \\ \colhline
 (E.g., Photosensors production)  &     &             \\ \colhline
 &     &             \\ \colhline 
 &     &             \\ \colhline
 &     &             \\ \colhline 
 &     &             \\ \colhline
 &     &             \\ \colhline
\rowcolor{dunepeach} DUNE FD Integration \& Installation  &  &     \\ \colhline
 &     &             \\ \colhline
 &     &             \\ \colhline 
 &     &             \\ \colhline
 &     &             \\ \colhline 
 &     &             \\ \colhline
 (last line) &     &             \\
\end{dunetable}
\end{verbatim}

This is a reference to Table~\ref{tab:XXcostsumm} (\verb|\ref{tab:XXcostsumm}|).

\begin{dunetable}
[Cost Summary]
{p{0.5\textwidth}p{0.2\textwidth}p{0.2\textwidth}}
{tab:XXcostsumm}
{Cost Summary}   
Cost Item & M\&S (k\$ US) & Labor Hours \\ \toprowrule
\rowcolor{dunepeach} Design, Engineering and R\&D &  &     \\ \colhline
 (E.g., Photosensors design) &     &             \\ \colhline
 (E.g., Mechanics design) &     &             \\ \colhline
 &     &             \\ \colhline
 &     &             \\ \colhline
 &     &             \\ \colhline
 &     &             \\ \colhline
\rowcolor{dunepeach} Production Setup &  &     \\ \colhline
 (E.g., Photosensors production setup)  &     &             \\ \colhline
 &     &             \\ \colhline 
 &     &             \\ \colhline
 &     &             \\ \colhline 
 &     &             \\ \colhline
 &     &             \\ \colhline
\rowcolor{dunepeach} Production &  &     \\ \colhline
 (E.g., Photosensors production)  &     &             \\ \colhline
 &     &             \\ \colhline 
 &     &             \\ \colhline
 &     &             \\ \colhline 
 &     &             \\ \colhline
 &     &             \\ \colhline
\rowcolor{dunepeach} DUNE FD Integration \& Installation  &  &     \\ \colhline
 &     &             \\ \colhline
 &     &             \\ \colhline 
 &     &             \\ \colhline
 &     &             \\ \colhline 
 &     &             \\ \colhline
 (last line) &     &             \\
\end{dunetable}

%%%%%%%%%%%%%%%%%%%%%%%%%%%%%%%
\subsection{Schedule/Milestone Tables}
\label{sec:tables-sched}

This is a standard table template for the TDR schedules.  It contains overall FD dates from Eric James as of March 2019 (orange) that are held in macros so that the TDR team can change them if needed. Please do not edit these lines! Please add your milestone dates to fit in with the overall FD schedule. 

\begin{verbatim}
\begin{dunetable}
[Consortium X Schedule]
{p{0.65\textwidth}p{0.25\textwidth}}
{tab:Xsched}
{Consortium X Schedule}   
Milestone & Date (Month YYYY)   \\ \toprowrule
Technology Decision Dates &      \\ \colhline
Final Design Review Dates &      \\ \colhline
Start of module 0 component production for ProtoDUNE-II &      \\ \colhline
End of module 0 component production for ProtoDUNE-II &      \\ \colhline
\rowcolor{dunepeach} Start of \dword{pdsp}-II installation& \startpduneiispinstall      \\ \colhline
\rowcolor{dunepeach} Start of \dword{pddp}-II installation& \startpduneiidpinstall      \\ \colhline
 \dword{prr} dates &      \\ \colhline
Start of  (component 1) production  &      \\ \colhline
Start of (component 2) production  &      \\ \colhline
Start of  (component 3) production  &      \\ \colhline
\rowcolor{dunepeach}South Dakota Logistics Warehouse available& \sdlwavailable      \\ \colhline
\rowcolor{dunepeach}Beneficial occupancy of cavern 1 and \dword{cuc}& \cucbenocc      \\ \colhline
\rowcolor{dunepeach} \dword{cuc} counting room accessible& \accesscuccountrm      \\ \colhline
\rowcolor{dunepeach}Top of \dword{detmodule} \#1 cryostat accessible& \accesstopfirstcryo      \\ \colhline
End of  (component 1) production  &      \\ \colhline
... & ...                       \\ \colhline
\rowcolor{dunepeach}Start of \dword{detmodule} \#1 TPC installation& \startfirsttpcinstall      \\ \colhline
\rowcolor{dunepeach}End of \dword{detmodule} \#1 TPC installation& \firsttpcinstallend      \\ \colhline
\rowcolor{dunepeach}Top of \dword{detmodule} \#2 accessible& \accesstopsecondcryo      \\ \colhline
 \rowcolor{dunepeach}Start of \dword{detmodule} \#2 TPC installation& \startsecondtpcinstall      \\ \colhline
\rowcolor{dunepeach}End of \dword{detmodule} \#2 TPC installation& \secondtpcinstallend      \\ \colhline

last item & ...                         \\
\end{dunetable}

\end{verbatim}

This is a reference to Table~\ref{tab:Xsched} (\verb|\ref{tab:Xsched}|).

\begin{dunetable}
[Consortium X Schedule]
{p{0.65\textwidth}p{0.25\textwidth}}
{tab:Xsched}
{Consortium X Schedule}   
Milestone & Date (Month YYYY)   \\ \toprowrule
Technology Decision Dates &      \\ \colhline
Final Design Review Dates &      \\ \colhline
Start of module 0 component production for ProtoDUNE-II &      \\ \colhline
End of module 0 component production for ProtoDUNE-II &      \\ \colhline
\rowcolor{dunepeach} Start of \dword{pdsp}-II installation& \startpduneiispinstall      \\ \colhline
\rowcolor{dunepeach} Start of \dword{pddp}-II installation& \startpduneiidpinstall      \\ \colhline
 \dword{prr} dates &      \\ \colhline
Start of  (component 1) production  &      \\ \colhline
Start of (component 2) production  &      \\ \colhline
Start of  (component 3) production  &      \\ \colhline
\rowcolor{dunepeach}South Dakota Logistics Warehouse available& \sdlwavailable      \\ \colhline
\rowcolor{dunepeach}Beneficial occupancy of cavern 1 and \dword{cuc}& \cucbenocc      \\ \colhline
\rowcolor{dunepeach} \dword{cuc} counting room accessible& \accesscuccountrm      \\ \colhline
\rowcolor{dunepeach}Top of \dword{detmodule} \#1 cryostat accessible& \accesstopfirstcryo      \\ \colhline
End of  (component 1) production  &      \\ \colhline
... & ...                       \\ \colhline
\rowcolor{dunepeach}Start of \dword{detmodule} \#1 TPC installation& \startfirsttpcinstall      \\ \colhline
\rowcolor{dunepeach}End of \dword{detmodule} \#1 TPC installation& \firsttpcinstallend      \\ \colhline
\rowcolor{dunepeach}Top of \dword{detmodule} \#2 accessible& \accesstopsecondcryo      \\ \colhline
 \rowcolor{dunepeach}Start of \dword{detmodule} \#2 TPC installation& \startsecondtpcinstall      \\ \colhline
\rowcolor{dunepeach}End of \dword{detmodule} \#2 TPC installation& \secondtpcinstallend      \\ \colhline

last item & ...                         \\
\end{dunetable}

%%%%%%%%%%%%%%%%%%%%%%%%%%%%%%%%%%%%%%%%%%%%%%%%%%%%%%%%%%%%%%%%%%%%
\section{Labels and Intra-document References}
\label{sec:latex-intra-doc-ref}

Assume that any chapter, section or important sub-, subsub- section
or any figure or table environment may need to be referenced
elsewhere in the text. 

Just below a chapter heading and any significant section heading a
label should be added so it can be referenced. Use the defined label in a \verb|\ref{mylabel}| in order to make reference
to the chapter, section, figure, etc.

For example:

\begin{verbatim}
\chapter{A Chapter}
\label{ch:a-chapter}

\section{A Section}
\label{sec:a-section}

\subsection{A Subsection}
\label{sec:a-subsection}

... as described in Chapter~\ref{ch:a-chapter} ... 
or Section~\ref{sec:a-section} ... 
or Section~\ref{sec:a-subsection} ...

\end{verbatim}

When you reference a chapter, section, subsection, figure, table,
etc., capitalize the word ``Chapter'' or whatever it is, e.g., ``as
shown in Section~\ref{sec:tech-mainfile}.''
Use the word ``Section'' even if it's a subsection or subsubsection,
and use the tilde sign to keep the number on the same line as the word
that precedes it.

For figures, the labels and references need to include ``fig:'' before the actual label name, as mentioned in Section~\ref{sec:latex-figures}; for tables, ``tab:'' must be prepended to the label name (see Section~\ref{sec:latex-tables}).


%%%%%%%%%%%%%%%%%%%%%%%%%%%%%%%%%%%%%%%%%%%%%%%%%%%%%%%%%%%%%%%%%%%%
\section{Citations and the Bibliography}
\label{sec:latex-cit}

%%%%%%%%%%%%%%%%%%%%%%%%%%%%%%%%%%%%
\subsection{The Bibliography File Containing the Citations}
\label{sec:latex-bib-file}


For the TDR, we have created the file \texttt{common/tdr-citedb.bib} to contain all the citations.
Please email all the bibliography entries that you need to Luke Corwin (Luke.Corwin@sdsmt.edu); he is ensuring that they all make it into the github repository.

Before requesting a new entry to \texttt{tdr-citedb.bib}, please search through it
to see whether the entry you wish to add is already there.
%When adding citations to this file, if possible, copy the BibTeX rendering of the citation from \url{http://inspirehep.net}.  Except...

When possible, send Luke the BibTeX rendering of your needed citation from \url{http://inspirehep.net}.  Except...

For any documents in docdb, just send him the docdb numbers. Please use the form %defined in Section~\ref{sec:tables-intfc}, namely: 
\verb|bib:docdbNNNN| for the label in your text; Luke will format the actual reference properly.  %For interface documents, you MUST use this form.

% Please read the guidelines at the beginning of that file. We have left a well-populated \texttt{common/citedb.bib} in the TDR directory in case you want to search for a citation that may have appeared in the CDR, or other DUNE/LBNF document.
(For the TDR you can stop reading this section here.)

Note that the bib file is \textbf{not} in \LaTeX{} format and in particular does not
indicate comments via \texttt{\%} characters. (Again, guidelines are in the file itself.)
The generated bibliography reflects the order in which citations are referenced in the text, not the order of entries in this file.

%Manual care must be taken to avoid duplication. %\fixme{BV found something; I have to look at it. AH}

JabRef (\url{http://www.jabref.org/}) is a cross-platform Java GUI that can be used to search bibliographies, possibly investigating duplicities, by working on any loaded .bib file.

%%%%%%%%%%%%%%%%%%%%%%%%%%%%%%%%%%%%
\subsection{Referencing Citations}
\label{sec:latex-ref}

Referencing citations is done like \verb|\cite{strunk}| which gives \cite{strunk}.
To reference multiple citations at the same place, use \verb|\cite{strunk,ref2,ref3}| which gives  [1,2,3].

(Compiling the bibliography entries into the document requires an extra step: run ``bibtex'' on
 guidance.tex, then run pdflatex on it again a couple of times. Otherwise you'll see [?] here 
 and no bibliography entry at the end.) 
The key \texttt{strunk} matches an entry in the \texttt{common/tdr-citedb.bib}
file (as relative to the top-level directory).

%%%%%%%%%%%%%%%%%%%%%%%%%%%%%%%%%%%%%%%%%%%%%%%%%%%%%%%%%%%%%%%%%%%%
\section{Vendors and Trademarks}
\label{sec:trademarks}

Please only reference specific vendors when the particular vendor choice is significant. 
We should use a consistent scheme for these. 
 
Accompany the first mention of a commercial product with a footnote.  This will improve readability by allowing for quick identification of the commercial term in question. 
Do not use the trademark symbol \texttrademark{} or registration mark \textregistered{} in the text, but use it in the footnote where appropriate. Add the footnote only on the first occurrence of the product name in a chapter.
Most commercial product names should be capitalized.
Use a short version of the commercial product name in the text itself.
 
The structure of the footnote should be the following:
Begin with a short descriptive phrase.  Then give the name of the company.  Add \texttrademark{} or  \textregistered{} according to what the company does on their web site. Then provide the company's top-level URL. If an electronic address does not exist, use a physical mailing address. Example:

In your text, write ``The component installation requires one Widgetmaster \\
2000\verb|\footnote{Widgetmastermaker\texttrademark{} Widgetmaster 2000 iron bending machine, | \\
\verb|Widgetmastermaker\texttrademark{} Power Widgets http:widgetmastermaker.com.}| because ...''

%%%%%%%%%%%%%%%%%%%%%%%%%%%%%%%%%%%%%%%%%%%%%%%%%%%%%%%%%%%%%%%%%%%%
\section{Standard \LaTeX{} Sectioning, ``The DUNE Way''!}
\label{sec:latex-sectioning}

Most documents get subdivided into chapters (for larger documents) and/or sections and subsections. Please subdivide according to these standards:

\begin{itemize}
\item A subdivision of a larger portion should have content that relates to some aspect of the larger portion. 
\item  If you create one subdivision, create at least one more. Otherwise, the topic of your one subdivision is (by definition) the same as that of the larger portion.
\item In your \LaTeX{} source, add lines of percent signs (comments) to make it easy to find where sections begin and end, as illustrated in the source for this file. (This really helps the editor!)
\end{itemize}

\begin{verbatim}
%%%%%%%%%%%%%%%%%%%%%%%%%%%%%%%%%%%%%%%%%%%%%%%%%%%%%%%%%%%%%%%%%%%%
\end{verbatim}

Please capitalize all significant words in a heading (this is a change from 
previous DUNE documents).

The following sectioning macros are available, ordered from bigger to smaller:

\begin{verbatim}
\chapter{A Chapter}
\section{A Section}
\subsection{A Subsection}
\subsubsection{A Subsubsection}
\end{verbatim}

We recommend that you stop the subsectioning at this point, but you can go down a couple of levels further.
%Consult with the technical editors if you feel finer grained sectioning is required.
Starting from \verb|\subsection|, this produces the following:

%%%%%%%%%%%%%%%%%%%%%%%%%%%%%%%%%%%%%
\subsection{A Subsection}
\label{sec:latex-sec-sub}

This is a subsection.

%%%%%%%%%%%%%%%%%%%
\subsubsection{A Subsubsection}
\label{sec:latex-sec-subsub}

This is a subsubsection.

%%%%%%%%%%%%%%%%%%%
\subsubsection{A Second Subsubsection}
\label{sec:latex-sec-subsub2}

Remember, if you have one, you need at least one more.

%%%%%%%%%%%%%%%%%%%%%%%%%%%%%%%%%%%%%
\subsection{A Second Subsection}
\label{sec:latex-sub2}

Ditto.
